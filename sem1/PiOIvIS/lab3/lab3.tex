\documentclass[10pt,twocolumn]{article}

\usepackage[pdftex]{graphicx}
\usepackage{enumitem}
\usepackage{indentfirst}
\usepackage{ledmac}

\setlist{nolistsep}
\pagenumbering{arabic}
\hoffset=0in  
\voffset=-1in

\setlength{\columnwidth}{5in}
\setlength{\columnsep}{0.30in}
\setlength{\textheight}{9in}

\renewcommand{\thesection}{\Roman{section}}
\setcounter{page}{212}

\begin{document}
\small
{\setlength{\parindent}{0in} assumptions. When forming an ontology of a subject
domain based on one of the specified top-level ontologies,
it can be more easily integrated with other subject
ontologies. The problem is that there are quite a lot
of top-level ontologies and giving preference to one of
them becomes a certain search problem that requires a
lot of time and effort. In addition, some of them do not
have open access and are also badly compatible with the
semantic Web.}


The ISO 15926 ontology [17] is considered separately.
This standard is not only a top-level ontology but also
a thesaurus of the processing industry, including the
structure of retention and access to the ontological base.
Standardization is implemented by using well-defined
templates for technical and operational information, that
include classes and relations of the invariant and temporal
parts of the ontology. The advantages of this ontological
model are the typification and identification of data located on the Internet; information is stored in RDF-triplets,
access to triplet storages occurs using the SPARQL query
language, etc. When creating this model, the developers
tried to cover all aspects of requests that may arise
in manufacturing. As a result, the model has hundreds
of nested classes and attributes at the lower levels of
production (description of technological equipment), most
of which may not be used in practice. The temporal part
increases the complexity of the model several times.


Thus, the ontology according to the ISO 15926 standard
is most suitable for the specified problem. However, it
should be noted that, taking into account the need for
a common equipment and ISA-88 knowledge base, it
was decided to implement the equipment knowledge base
using OSTIS. In addition, there are restrictions in the
ISO 15926 standard that are not present in the OSTIS
Technology.

\vspace{5mm}
\beginnumbering
\pstart\vfill\noindent\normalsize\leftskip=3mm\rightskip=3mm\sc
IV. About ISA-88 and the criteria for
\pend
\pstart\vfill\noindent\normalsize\leftskip=3mm\rightskip=3mm\sc
the decomposition of technology and
\pend
\pstart\vfill\noindent\normalsize\centering\sc
equipment
\pend
\endnumbering
\vspace{5mm}

As already noted above, when creating the equipment
hierarchy, an engineer faces a number of problems related
to the need to take into account many factors. The larger
the technological scheme in terms of the amount of
equipment and the more connectivities it has, the more
difficult it is to allocate logically related equipment in it.
The difficulty also lies in the fact that the standards do not
and cannot have all the criteria for allocation. Therefore,
this problem should be considered both from the point
of view of the limitations and functional requirements
of the ISA-88 standard and from the point of view of
experience from best practices. Both can be put into the
knowledge base [18].
 
 
To begin with, let us highlight clear restrictions, using
which it is quite easy to determine whether the equipment
belongs to one of the hierarchy levels. According to ISA-88, these levels are:
\vspace{0.9mm}
\begin{enumerate}
\setlength\itemsep{0em}
\item[1)] the level of the process cells;
\item[2)] the level of units;
\item[3)] the level of equipment modules;
\item[4)] the level of control modules.
\end{enumerate}
\vspace{0.9mm}


According to the standard, “a process cell is a logical
grouping of special equipment that includes the equipment
required for production of one, or more batches”. Exactly
from the point of view of batch production, the process
cell is distinguished. If the entire batch of semi-products
is not developed in the framework of the process cell,
the equipment that is needed for this should be included.
Within one process cell, there may be several connected
elements of special equipment capable of producing several
batches in parallel. If batches cannot be separated, they
remain within the same process cell. In addition, the
process cell must contain at least one unit. 


The allocation of units is a little less obvious. There
are several clear criteria:
\vspace{0.9mm}
\begin{enumerate}
\setlength\itemsep{0em}
\item[1)] one unit cannot contain several batches;

\item[2)] each technological action occurs immediately (simultaneously) with all the material within one unit;
\item[3)] the technological operation begins and ends within
the same unit.
\end{enumerate}
\vspace{0.9mm}
Even less obvious conditions for choosing and combining are the statements:
\vspace{0.9mm}
\begin{itemize}
\setlength\itemsep{0em}
\item the unit can include all the equipment and control
modules involved in technological actions;
\item the unit can work with part of the batch.
\end{itemize}
\vspace{0.9mm}


All equipment, except for the control module, can
implement procedural control. That is, from the point
of view of technology, it contains some procedural elements that perform a technological operation, separating
itself from the method of its implementation. There are
operational directives, for example, “heat to the required
temperature”, as opposed to the directives “open valve
1” or “if TE101 $>$ 23, close the valve”. The last control
directive refers to equipment, not technology, and is called
“basic control” in the standard. This is the main criterion
that determines the principle of allocating the control
module – this equipment does not contain procedural
control. In addition, this part of the hierarchy enables
real interaction with concrete equipment, while the other
levels are more role groups. Therefore, the level of control
modules cannot be omitted in the hierarchy.


The concept of procedural control per se is also not
clear enough. It is difficult to formalize it as well as to
define in an ontology. However, according to the standard,
there are certain features inherent in it, in contrast to the
basic control, such as visibility at the recipe level, a
characteristic state engine, abstraction from equipment,
etc.


As for the control module, there is one indirect but
very useful property as a selection criterion  $-$  this
type of equipment is shown in the P\&ID-schemes as
instrumentation. According to the standard, the control
\end{document}